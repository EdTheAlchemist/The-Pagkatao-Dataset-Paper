%
% File coling2020.tex
%
% Contact: feiliu@cs.ucf.edu & liang.huang.sh@gmail.com
%% Based on the style files for COLING-2018, which were, in turn,
%% Based on the style files for COLING-2016, which were, in turn,
%% Based on the style files for COLING-2014, which were, in turn,
%% Based on the style files for ACL-2014, which were, in turn,
%% Based on the style files for ACL-2013, which were, in turn,
%% Based on the style files for ACL-2012, which were, in turn,
%% based on the style files for ACL-2011, which were, in turn, 
%% based on the style files for ACL-2010, which were, in turn, 
%% based on the style files for ACL-IJCNLP-2009, which were, in turn,
%% based on the style files for EACL-2009 and IJCNLP-2008...

%% Based on the style files for EACL 2006 by 
%%e.agirre@ehu.es or Sergi.Balari@uab.es
%% and that of ACL 08 by Joakim Nivre and Noah Smith

\documentclass[11pt]{article}
\usepackage{coling2020}
\usepackage{times}
\usepackage{url}
\usepackage{latexsym}



%\setlength\titlebox{5cm}
\colingfinalcopy % Uncomment this line for the final submission

% You can expand the titlebox if you need extra space
% to show all the authors. Please do not make the titlebox
% smaller than 5cm (the original size); we will check this
% in the camera-ready version and ask you to change it back.


\title{The PagkataoKo Dataset: A Multimodal Dataset of Filipino Instagram and Twitter Users for Automatic Personality Recognition}

\author{Edward Tighe\thanks{~The corresponding author.} , Luigi Acorda , Alexander Agno , Jesah Gano , \\ 
{\bf Timothy Go , Gabriel Santiago \and Claude Sedillo} \\
  College of Computer Studies, De La Salle University \\
  2401 Taft Avenue, Malate, Manila, Philippines \\
  {\tt \{edward.tighe, luigi\_acorda, alexander\_agno, jesah\_gano,} \\
  {\tt timothy\_go, gabriel\_santiago, claude\_sedillo\}@dlsu.edu.ph} \\}

\date{}

\begin{document}
\maketitle
\begin{abstract}
  This document contains the instructions for preparing a paper submitted
  to COLING-2020 or accepted for publication in its proceedings. The document itself
  conforms to its own specifications, and is therefore an example of
  what your manuscript should look like. These instructions should be
  used for both papers submitted for review and for final versions of
  accepted papers. Authors are asked to conform to all the directions
  reported in this document.
\end{abstract}

%
% The following footnote without marker is needed for the camera-ready
% version of the paper.
% Comment out the instructions (first text) and uncomment the 8 lines
% under "final paper" for your variant of English.
% 
\blfootnote{
    %
    % for review submission
    %
    %\hspace{-0.65cm}  % space normally used by the marker
    Place licence statement here for the camera-ready version. See
    Section~\ref{licence} of the instructions for preparing a
    manuscript.
    %
    % % final paper: en-uk version 
    %
    % \hspace{-0.65cm}  % space normally used by the marker
    % This work is licensed under a Creative Commons 
    % Attribution 4.0 International Licence.
    % Licence details:
    % \url{http://creativecommons.org/licenses/by/4.0/}.
    % 
    % % final paper: en-us version 
    %
    % \hspace{-0.65cm}  % space normally used by the marker
    % This work is licensed under a Creative Commons 
    % Attribution 4.0 International License.
    % License details:
    % \url{http://creativecommons.org/licenses/by/4.0/}.
}

\section{Introduction} %WHY IS THE WORK IMPORTANT
\label{intro}
% APR, what it is and (what is personality?)
Automatic Personality Recognition (APR) is task under Personality Computing that deals with the recognition of one's true (self-assessed) personality traits \cite{vinciarelli2014survey}. APR is rooted on the idea that personality is externalized and embedded in to one's detectable behaviour -- whether it be through writing \cite{vinciarelli2014survey} NEED BETTER CITATIONS HERE.

% APR w/ social media
While APR has been explored in many different domains, a good number of recent work focus on social media sites (SNSs).

% Current issues 

% What our paper addresses

% contributions

the task of inferring an individual's self-assessed personality traits based on data produced from or by the individual \cite{vinciarelli2014survey}. 

\section{Review of Related Literature}

\section{Collection Methodology}

\section{Dataset Characteristics}

\subsection{Demographics}

\subsection{Personality Trait Data}

\subsection{}

% include your own bib file like this:
\bibliographystyle{coling}
\bibliography{coling2020}

\end{document}
